\documentclass[../rapport.tex]{subfiles}

\begin{document}
\subsection{Extension 4 - Analyse statistique de la consommation énergétique - Gilles \textsc{Jaunart}}

\subsubsection{Description de l'extension :}

L'objectif de cette extension est de permettre au client de visualiser les données de consommation énergétique associées à son portefeuille. En plus de cela, l'extension offre une interface visuelle permettant d'analyser des statistiques relatives à la consommation énergétique. 
Les statistiques fournies au client comprennent le minimum, le maximum, la moyenne et l'écart-type de la consommation énergétique. D'autres données telles que la médiane et les quartiles ont été exclues, car elles ont été jugées soit inutiles, soit trop difficiles à interpréter pour un client non spécialisé. Il convient de noter que des infobulles sont disponibles pour expliquer la signification de la moyenne et de l'écart-type. 

\subsubsection{Vidéo de présentation des fonctionnalités} 
La vidéo de présentation de l'extension se trouve dans la playlist suivante : \href{https://www.youtube.com/playlist?list=PLoqkbU5JK0C9hzut7JvQ1AznLSYIX3d5O}{\textit{https://www.youtube.com/playlist?list=PLoqkbU5JK0C9hzut7JvQ1AznLSYIX3d5O}}. \\ \\

\subsubsection{Avantages et inconvénients} 
Mon application dispose d'un beau système de graphique qui permet de visualiser les données de manière claire et concise. De plus, l'application est très réactive et permet de visualiser les données de manière dynamique. Malgré une implémentation back-end des fonctions de récupération des données présente, l'application n'a pas pu être finalisée en raison d'une contrainte de temps et les données utilisées sont codées en dure.

\subsubsection{Fonctionnalités manquantes} 

Malgré la découverte de méthodes permettant de détecter les problèmes, tels que l'utilisation de la méthode des outliers pour repérer des comportements inhabituels dans la consommation d'énergie, tels qu'une fuite d'eau ou de gaz, ces méthodes n'ont pas été mises en œuvre en raison d'une contrainte de temps. 
De plus, la comparaison de la consommation entre des habitations ayant des caractéristiques similaires n'a pas été réalisée conformément aux demandes. Au lieu de cela, une approche très basique a été utilisée, qui ne prend pas en compte les caractéristiques comparables des habitations. La comparaison est effectuée de manière globale et ne tient pas compte des spécificités de chaque logement.

\end{document}