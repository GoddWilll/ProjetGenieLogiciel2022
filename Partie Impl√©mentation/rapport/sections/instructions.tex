\documentclass[../rapport.tex]{subfiles}

\begin{document}


\subsection{Instructions d'installation}

Pour tester l'application sur votre ordinateur, il est nécessaire de posséder les outils suivants : 

\begin{itemize}
    \item Nodejs (version 18 ou supérieure)
    \item Gradle (version 7.5.1 ou supérieure)
    \item Java (version 17 ou supérieure)
\end{itemize}

Pour lancer l'application, il faudra dans un premier temps se rendre dans le dossier \textit{back\_end}. Une fois dans ce dossier, il faudra lancer la commande suivante : 

\begin{lstlisting}
  gradle bootRun
\end{lstlisting}

pour démarrer le serveur. Celui ci va démarrer sur le port 8080. Si une autre application utilise ce port, l'application ne pourra pas se lancer.

Une fois le serveur démarré, il faudra se rendre dans le dossier \textit{front\_end}. Une fois dans ce dossier, il faudra se rendre dans un des sous dossiers, parmi lesquels : \textit{client\_app\_commune}, \textit{client\_app\_extension\_facturation} et toutes les autres extensions représentées par leurs noms respectifs.

Une fois dans un de ces sous dossiers, il faudra lancer la commande suivante : 

\begin{lstlisting}
  npm install
\end{lstlisting}

Cette commander permettra d'installer toutes les dépendances nécessaires au bon fonctionnement de l'application client. Une fois les dépendances installées, il faudra lancer l'application avec la commande : 

\begin{lstlisting}
  npm start
\end{lstlisting}

L'application se lancera sur le port 3000. Si une autre application utilise ce port, l'application ne pourra pas se lancer.

\end{document}